\documentclass[a4paper,10pt]{article}

\title{How do I code?}
\author{Student Robotics}

\begin{document}

\maketitle

\hfill ``I don't know how to program!''\\
% Create a box for hfill space to get stuck to
\hphantom{fgasdf}\hfill --- Many SR competitors\\
\\
``Everyone can program!''\\
--- Many other people\\

\section{Introduction}

This is a guide on how to go about programming your robot, as part of the
Student Robotics competition. Writing the \textit{text} of your program is
covered elsewhere, for example the Python
tour\footnote{https://www.studentrobotics.org/docs/tutorials/python} --- this
guide is about how to \textit{design} your program. We only assume that you
know four things:
\begin{enumerate}
	\item Variables and assignment
	\item Conditionals (``if'' statements)
	\item Loops
	\item Functions
\end{enumerate}
All of which are covered by the Python tour.


\begin{enumerate}
\item There's a conflict in what people say
\item Differentiate syntax and application implementation
\item Identify the state space of all the things as being the problem
\item Say that this document describes a technique to get a handle on that
\end{enumerate}

\section{What's the point of programming?}
\begin{enumerate}
\item Mission statement of the robot
\item Replicates decision process
\item Illustrate example motions / arena path / whatever
\item What's available to achieve this are inputs / outputs
\item As we discussed, state space is huge
\item What you want is a small program that achieves this all, and:
\begin{itemize}
	\item Can be well understood by the reader
	\item Can be easily modified to do something different
	\item Is robust (against unexpected things happening)
\end{itemize}
\end{enumerate}

\section{Design patterns and state machines}
\begin{enumerate}
\item Programmers re-use lots of things. So should you.
\item \textit{Design patterns} exist: common ways of thinking about things
\item In the context of robotics, the \textit{state machine} is the most common
\item There are thousands in your phone
\item They have some exellent attributes:
\begin{itemize}
	\item Very limited number of states
	\item Clear reasons for a decision being made
	\item Easy to debug.
	\item You can compose them
\end{itemize}
\item High level view of what states and transitions may be
\end{enumerate}

\section{A simple example}
\begin{enumerate}
\item Formally show some states and transitions
\item Say that we're going to make a state machine for the example path above
\item Do so
\item Illustrate path with states / transitions, actually draw dot file?
\item Show example code (python), explain syntax isn't important
\end{enumerate}

\section{Making it robust}
\begin{enumerate}
\item Self loops if things go wrong
\item Think about valid error states and what to do
\item Now consider full state space: what would you do?
\item Ideally you consider all cases; just try for good coverage though
\item Good code practice (functions etc)
\end{enumerate}

\section{Inputs and outputs}
\begin{enumerate}
\item Identify loop cycle of a state machine
\item Read inputs and states, decide outputs, wait time, repeat
\item Not sure what else
\end{enumerate}

\section{Summary}
fgasdf
\end{document}
